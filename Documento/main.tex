\documentclass[fleqn,10pt]{SelfArx} 

\usepackage[utf-8]{babel} 
%----------------------------------------------------------------------------------------
%	COLUMNS
%----------------------------------------------------------------------------------------

\setlength{\columnsep}{0.55cm} 
\setlength{\fboxrule}{0.75pt} 
\usepackage[english]{babel} % Specify a different language here - english by default
\usepackage{lipsum} % Required to insert dummy text. To be removed otherwise
%----------------------------------------------------------------------------------------
%	COLORS
%----------------------------------------------------------------------------------------

\definecolor{color1}{RGB}{0,0,90} % Color of the article title and sections
\definecolor{color2}{RGB}{230, 231, 231} % Color of the boxes behind the abstract and headings

%----------------------------------------------------------------------------------------
%	HYPERLINKS
%----------------------------------------------------------------------------------------

\usepackage{hyperref} % Required for hyperlinks

\hypersetup{
	hidelinks,
	colorlinks,
	breaklinks=true,
	urlcolor=color2,
	citecolor=color1,
	linkcolor=color1,
	bookmarksopen=false,
	pdftitle={Title},
	pdfauthor={Author},
}

%----------------------------------------------------------------------------------------
%	ARTICLE INFORMATION
%----------------------------------------------------------------------------------------

\JournalInfo{Método de Muller, 2022} % Journal information

\PaperTitle{Método de Muller} % Article title

\Authors{Juan Páez\textsuperscript{1}, Santiago Zuñiga\textsuperscript{2}} % Authors
\affiliation{\textsuperscript{1}\textit{Facultad de ingeniería, Pontificia Universidad Javeriana}} % Author affiliation
\affiliation{\textsuperscript{2}\textit{Facultad de ingeniería, Pontificia Universidad Javeriana}} % Author affiliation
\affiliation{\textbf{Autor correspondiente}: jd.paez@javeriana.edu.co} % Corresponding author

\Keywords{Muller --- Complex Roots --- Error --- Solution --- One Variable --- POSIBLES PALABRAS CLAVE} % Keywords - if you don't want any simply remove all the text between the curly brackets
\newcommand{\keywordname}{Keywords} % Defines the keywords heading name

%----------------------------------------------------------------------------------------
%	ABSTRACT
%----------------------------------------------------------------------------------------

\Abstract{In this paper we will explain the Muller method used 
to solve equations in one variable with complex roots, we will also 
present an application problem with the computational solution and 
the respective error analysis.}

%----------------------------------------------------------------------------------------

\begin{document}

\maketitle % Output the title and abstract box

\thispagestyle{empty} % Removes page numbering from the first page

%----------------------------------------------------------------------------------------
%	ARTICLE CONTENTS
%----------------------------------------------------------------------------------------

\section*{Introducción} % The \section*{} command stops section numbering

\addcontentsline{toc}{section}{Introduction} % Adds this section to the table of contents
A la hora de solucionar ecuaciones no lineales en una variable, 
es decir encontrar o aproximar sus raíces, se pueden utilizar 
distintos métodos, como el despeje directo, Newton, posición falsa, 
entre otros. Sin embargo, se presenta un problema cuando las funciones que 
queramos evaluar tengan raíces complejas, ya que los métodos mencionados
no son los indicados para este tipo planteamientos. 
Es por eso que el matemático estadounidense, David Eugene \textbf{Muller}, 
propuso un método para poder obtener un aproximación de este tipo de 
funciones el cual será explicado y desarrollado a lo largo de este documento.


%------------------------------------------------

\section{Definición}

El método de Muller, esta muy relacionado con el método de la secante, 
teniendo en cuenta que este consiste en tener una aproximación de la 
raíz a partir de dos puntos en la función $F(x)$. En el caso de 
Muller consiste en tener 3 puntos sobre la gráfica de la función, 
siendo estos una composición cuadrática, la cual da una aproximación de la raíz compleja de $F(x)$.
\\\
\textbf{DE AQUI EN ADELANTE ES DE LA PLANTILLA}
\\\
\subsection{Subsection}

\lipsum[6] % Dummy text

\paragraph{Paragraph} \lipsum[7] % Dummy text
\paragraph{Paragraph} \lipsum[8] % Dummy text

\subsection{Subsection}

\lipsum[9] % Dummy text

\begin{figure}[ht]\centering
	\includegraphics[width=\linewidth]{results}
	\caption{In-text Picture}
	\label{fig:results}
\end{figure}

Reference to Figure \ref{fig:results}.

%------------------------------------------------

\section{Results and Discussion}

\lipsum[10] % Dummy text

\subsection{Subsection}

\lipsum[11] % Dummy text

\begin{table}[hbt]
	\caption{Table of Grades}
	\centering
	\begin{tabular}{llr}
		\toprule
		\multicolumn{2}{c}{Name} \\
		\cmidrule(r){1-2}
		First name & Last Name & Grade \\
		\midrule
		John & Doe & $7.5$ \\
		Richard & Miles & $2$ \\
		\bottomrule
	\end{tabular}
	\label{tab:label}
\end{table}

\subsubsection{Subsubsection}

\lipsum[12] % Dummy text

\begin{description}
	\item[Word] Definition
	\item[Concept] Explanation
	\item[Idea] Text
\end{description}

\subsubsection{Subsubsection}

\lipsum[13] % Dummy text

\begin{itemize}[noitemsep] % [noitemsep] removes whitespace between the items for a compact look
	\item First item in a list
	\item Second item in a list
	\item Third item in a list
\end{itemize}

\subsubsection{Subsubsection}

\lipsum[14] % Dummy text

\subsection{Subsection}

\lipsum[15-23] % Dummy text

%------------------------------------------------

\phantomsection
\section*{Acknowledgments} % The \section*{} command stops section numbering

\addcontentsline{toc}{section}{Acknowledgments} % Adds this section to the table of contents

So long and thanks for all the fish \cite{Figueredo:2009dg, Smith:2012qr}.

%----------------------------------------------------------------------------------------
%	REFERENCE LIST
%----------------------------------------------------------------------------------------

\phantomsection
\bibliographystyle{unsrt}
\bibliography{sample.bib}

%----------------------------------------------------------------------------------------

\end{document}